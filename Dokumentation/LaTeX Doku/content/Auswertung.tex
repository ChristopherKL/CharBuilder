\newpage
\section{Einleitung}
In diesem Dokument wird die Entwicklung der Applikation CharBuilder für das Wahlpflichtfach Mobile Applikationen dargestellt. Die Aufgabenstellung hat uns sowohl die Wahl der Plattform als auch die Art der App offengelassen. Da wir in der Vorlesung die Grundlagen der App-Entwicklung

\section{Stand der Technik}
Im nachfolgenden wird erläutert welche Technologien wir für die Erstellung der CharBuilder App verwendet
und warum wir uns für die entschieden haben. Hierbei haben wir den Text in die Unterkategorien Android(Betriebssystem), Kotlin(Programmiersprache), Gson(Json Java-Lib), Gradle(Build Tool), Android Studio(IDEA) aufgeteilt, dies sind die wichtigsten Technologien die wird nutzen um unsere App zu entwickeln.

\subsection{Android}
Wie in der Einleitung geschrieben haben wir uns für Android als Betriebssystem entschieden. Da verschiedene Version dessen auf unterschiedlichen Geräten verteilt sind
muss man sich als Entwickler darüber hinaus auch für ein Mindest-API-Level entscheiden. Dies stellt die niedrigste Android Version dar, unter welcher die App ausgeführt werden kann.
Bei der Wahl des API-Levels spielen verschiedene Faktoren eine Rolle, unter anderem wie viele der Android Geräte welche Version des Betriebssystems ausführen oder ob der Entwickler
Funktionen nutzen möchte die erst ab einer bestimmten Version verfügbar sind.\\
%includegraphics("bild api verteilung android")
Wir haben uns entschieden, unsere Applikation lediglich für Android Betriebssysteme der Version 5.0(API Level 21) oder höher zu entwickeln.
Die Version 5 ist bereits am 12.11.2014 veröffentlicht worden und bringt einen großen Wandel in der Benutzeroberfläche durch Googles Material Design.
Dieses ist an den Gestaltungsstil "Flat Design" angelehnt und minimalistisch gehalten. Die Entscheidung trafen wir aufgrund der weitreichenden 
Änderungen in dieser Version, des bereits 3 Jahre in der Vergangenheit liegenden Veröffentlichungsdatums und der breiten Verteilung von 80,7\% aller Android Geräte.
%\cite{https://de.statista.com/statistik/daten/studie/180113/umfrage/anteil-der-verschiedenen-android-versionen-auf-geraeten-mit-android-os/}.\\

\subsection{Kotlin}

Kotlin ist eine neue Programmiersprache von JetBrains aus dem Jahr 2016. Wie auch Java kompiliert Kotlin zu JVM Bytecode. Es lässt sich dadurch sehr gut in das bestehende Java Ökosystem einbinden und kann alle Java Bibliotheken verwenden. Seit dem 17. Mai 2017 ist Kotlin eine von Android offiziel unterstützte Sprache.\\
%\cite{https://10clouds.com/blog/kotlin-android/}
Beide Teammitglieder haben zuvor noch nicht mit Kotlin gearbeitet, konnten sich jedoch aufgrund der Ähnlichkeit zu Java schnell zurechtfinden und gute Erfahrungen sammeln. Die Programmiersprache bietet dem Programmierer viele Vorteile wodurch sie immer beliebter wird unter Androidentwicklern. Der weitreichenste Vorteil ist es  null-Referenzen zu verhindern, die Sprache bietet einem "Null Safety". Eine weitere Änderung ist auch die Möglichkeit Datenklassen zu erstellen, diese sind Klassen welche lediglich Daten halten und keinerlei Methoden selbst implementieren. Ein Beispiel dieser kann man im Kapitel Implementierung finden. Der Großteil der Appentwicklung geschah mit Kotlin v1.15 später wurde dann ein Update auf Kotlin v1.2 durchgeführt. Dies liegt an dem Entwicklungszyklus von Kotlin, so wurde immer mit dem aktuellsten stabilen Release gearbeitet.
%include graphics(kotlin findviewbyid bild christopher)
%\cite{https://kotlinlang.org/}

\subsection{Gson}

Gson ist eine Java Bibliothek zur Serialization und Deserialization. Sie wird genutzt um Java Objekte in JSON umzuwandeln oder auch JSON zu Java Objekten zu wandeln. Hierbei ist die Möglichkeit Generics zu verwenden äußerst wichtig, da diese in der CharBuilder Applikation mehrmals zum Einsatz kommt. Wir haben diese Bibliothek ausgesucht da sie uns bereits bekannt war von früheren Projekten und wir positive Erfahrungen gemacht haben. Desweiteren wird die Software bereits seit 2008 entwickelt und konnte seitdem durch viele Revisionen und Verbesserungen überzeugen. Für die App verwendet wir die Gson Version 2.8.2.
%\cite{https://github.com/google/gson}

\subsection{Gradle}

Gradle ist ein weit verbreitetes Build-Tool welches automatisiert arbeitet und unterstützung für mehrere verschiedene Sprachen bietet. Es wird typischerweise für Android Applikationen verwendet welche mit Android Studio entwickelt werden, da das Tool tief in die Entwicklungsumgebung integriert ist. Zu unserer Entscheidung diese Programm zu werden kann nicht viel gesagt werden, da es wie oben erwähnt bereits integriert war und eine alternative neben großem Mehraufwand keine Vorzüge geboten hätte.

\subsection{Android Studio}

Android Studio ist die offizielle Entwicklungsumgebung für native Androidprogrammierung. Es bietet neben den bekannten IDE Funktionen wie Syntax Highlighting, Autovervollständigung, Instant Run(Es wird nur der veränderte Teil neu kompiliert) und Debugger auch einen Android Emulator um verschiede Geräte und Android Version zu simulieren. Die IDE basiert auf IntelliJ's Softwareprodukten. Mit der Version 3.0.1 wurde die Entwicklungsumgebung für Kotlin angepasst, dies ist auch die Version welche wir zur Entwicklung unserer App verwendeten. Dabei liefert die neue Version ebenfalls ein Programm um Java-Code direkt in Kotlin-Code umzuwandeln, dies erleichtert dem Entwickler anfangs den Einstieg sollte im späteren verlauf aber nur selten genutzt werden um eine einheitliche Codequalität zu erreichen.

\section{Anforderungen}

\section{Architektur}

\section{Implementierung}

\section{Test und Usability}

\section{Zusammenfassung}